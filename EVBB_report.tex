\documentclass[]{article}
\usepackage{lmodern}
\usepackage{amssymb,amsmath}
\usepackage{ifxetex,ifluatex}
\usepackage{fixltx2e} % provides \textsubscript
\ifnum 0\ifxetex 1\fi\ifluatex 1\fi=0 % if pdftex
  \usepackage[T1]{fontenc}
  \usepackage[utf8]{inputenc}
\else % if luatex or xelatex
  \ifxetex
    \usepackage{mathspec}
  \else
    \usepackage{fontspec}
  \fi
  \defaultfontfeatures{Ligatures=TeX,Scale=MatchLowercase}
\fi
% use upquote if available, for straight quotes in verbatim environments
\IfFileExists{upquote.sty}{\usepackage{upquote}}{}
% use microtype if available
\IfFileExists{microtype.sty}{%
\usepackage{microtype}
\UseMicrotypeSet[protrusion]{basicmath} % disable protrusion for tt fonts
}{}
\usepackage[margin=1in]{geometry}
\usepackage{hyperref}
\hypersetup{unicode=true,
            pdftitle={Analysis of electric vehicle usage patterns in New Zealand},
            pdfauthor={Rafferty Parker and Ben Anderson (University of Otago)},
            pdfborder={0 0 0},
            breaklinks=true}
\urlstyle{same}  % don't use monospace font for urls
\usepackage{longtable,booktabs}
\usepackage{graphicx,grffile}
\makeatletter
\def\maxwidth{\ifdim\Gin@nat@width>\linewidth\linewidth\else\Gin@nat@width\fi}
\def\maxheight{\ifdim\Gin@nat@height>\textheight\textheight\else\Gin@nat@height\fi}
\makeatother
% Scale images if necessary, so that they will not overflow the page
% margins by default, and it is still possible to overwrite the defaults
% using explicit options in \includegraphics[width, height, ...]{}
\setkeys{Gin}{width=\maxwidth,height=\maxheight,keepaspectratio}
\IfFileExists{parskip.sty}{%
\usepackage{parskip}
}{% else
\setlength{\parindent}{0pt}
\setlength{\parskip}{6pt plus 2pt minus 1pt}
}
\setlength{\emergencystretch}{3em}  % prevent overfull lines
\providecommand{\tightlist}{%
  \setlength{\itemsep}{0pt}\setlength{\parskip}{0pt}}
\setcounter{secnumdepth}{5}
% Redefines (sub)paragraphs to behave more like sections
\ifx\paragraph\undefined\else
\let\oldparagraph\paragraph
\renewcommand{\paragraph}[1]{\oldparagraph{#1}\mbox{}}
\fi
\ifx\subparagraph\undefined\else
\let\oldsubparagraph\subparagraph
\renewcommand{\subparagraph}[1]{\oldsubparagraph{#1}\mbox{}}
\fi

%%% Use protect on footnotes to avoid problems with footnotes in titles
\let\rmarkdownfootnote\footnote%
\def\footnote{\protect\rmarkdownfootnote}

%%% Change title format to be more compact
\usepackage{titling}

% Create subtitle command for use in maketitle
\newcommand{\subtitle}[1]{
  \posttitle{
    \begin{center}\large#1\end{center}
    }
}

\setlength{\droptitle}{-2em}

  \title{Analysis of electric vehicle usage patterns in New Zealand}
    \pretitle{\vspace{\droptitle}\centering\huge}
  \posttitle{\par}
  \subtitle{Statistical Report}
  \author{Rafferty Parker and Ben Anderson (University of Otago)}
    \preauthor{\centering\large\emph}
  \postauthor{\par}
      \predate{\centering\large\emph}
  \postdate{\par}
    \date{Last run at: 2019-03-22 16:26:10}

\usepackage{booktabs}
\usepackage{longtable}
\usepackage{array}
\usepackage{multirow}
\usepackage{wrapfig}
\usepackage{float}
\usepackage{colortbl}
\usepackage{pdflscape}
\usepackage{tabu}
\usepackage{threeparttable}
\usepackage{threeparttablex}
\usepackage[normalem]{ulem}
\usepackage{makecell}
\usepackage{xcolor}

\begin{document}
\maketitle

{
\setcounter{tocdepth}{2}
\tableofcontents
}
\section{Introduction}\label{introduction}

The New Zealand government has set a target of increasing the number of
EVs in New Zealand to 64,000 by 2021{[}@TranspowerNewZealand2017{]}.
High penetration of EVs would cause EV recharging to contribute a
substantial portion of total electricity load. A report prepared for
lines companies Orion, Powerco and Unison by Concept Consulting Group
entitled ``Driving change - Issues and options to maximise the
opportunities from large-scale electric vehicle uptake in New Zealand''
predicts that if all current light private vehicles were electric,
annual residential electricity consumption would increase by
approximately 30\%, whereas if all vehicles including trucks were
electric, this would increase the total electricity consumption of New
Zealand by approximately 41\%{[}@ConceptConsulting2018{]}.

New Zealand's total electricity demand varies throughout the day, with
weekdays in particular having two distinct ``peaks''; one in the
morning, and one in the evening{[}@TranspowerNZ2015{]}. Providing the
electricity to meet these demand peaks is a costly and inefficient
process{[}@Kahn2018{]}. Concurrent electric vehicle (EV) charging,
especially in the early evening when many motorists return home, would
have the potential to negatively impact the operation of the grid
through drastically increasing peak loads {[}@Azadfar2015{]}, leading to
an increased cost of electricity due to the requirement of expensive
upgrades to the electricity grid{[}@stephenson\_smart\_2017{]}.

The Concept Consulting report considers different methods of EV charging
in its models. The assumption that most drivers would begin charging
immediately after returning home is referred to as ``passive'' charging,
while charging that is programmed (either by the driver or by an
external entity) to occur during off-peak periods is referred to as
``smart''. The key findings of the Concept Consulting report are as
follows.

Under a scenario whereby 57\% of the current private vehicle fleet were
EVs (corresponding to one EV per household): * If all were charged in a
passive fashion, New Zealand's peak electricity demand would increase by
approximately 3,000MW * If all were charged in a ``smart'' fashion,
there would be no increase in peak demand

This report extends the work done by Concept Consulting, but utilises
actual data collected from electric vehicles, as opposed to using models
based on current usage patterns of internal combustion engine (ICE)
vehicles. The intention of the report is to provide further insight into
the potential effects on the New Zealand electricity grid that may occur
with a dramatic increase in EVs, so that these may be planned for and
mitigated. It is also inspired by the
\href{https://assets.publishing.service.gov.uk/government/uploads/system/uploads/attachment_data/file/764270/electric-chargepoint-analysis-2017-domestics.pdf}{UK
DoT statistical report 2018}.

To provide these insights we focus on * Observed demand * Daily demand *
Charging duration * State of charge * Time charging begins

\section{Data information}\label{data}

\subsection{Background}\label{background}

The data used has been provided by ``Flip the Fleet'', a community
organisation that hopes to increase uptake of electric vehicles in New
Zealand. Flip the Fleet have been collecting data on electric vehicle
usage patterns, collected using Exact IOT Limited's
\href{https://flipthefleet.org/ev-black-box/}{blackbox recorder}, a
small electronic device that connects to the vehicle's internal computer
and sends detailed data about the battery health, consumption, speed,
etc.

The data used consisted of 1291881 data points from 44 vehicles over 8
months (April 2018 - January 2019). The recorder provided measurements
at 1 minute frequency of charging power and battery charge state.

Due to privacy considerations, the data is not publicly available.

\subsection{Initial cleaning}\label{initial-cleaning}

There were 6 additional vehicles in the data provided that had no
recorded charging occur. These were immediately discarded leaving 44
remaining vehicles.

Some instances of charging power greater than 120kW were recorded. These
were considered anomalies and discarded, as these exceed the capacity of
the highest charging stations available in New
Zealand{[}@ConceptConsulting2018{]}.

Instances of battery state of charge being greater than 100\% or less
than 0\% were also discarded.

\subsection{Definitions and preparation}\label{cleaning}

Charging data has been broadly separated into two separate categories,
``standard'' and ``fast''. Standard charging is when the charger is
reading less than 7kW - this is considered the upper limit of what can
be obtained from a standard home charging scenario without an expensive
wiring upgrade{[}@ConceptConsulting2018{]}. Fast charging is all
charging above 7kW, and would likely occur at designated and
purpose-built fast charging stations.

The data was also categorised according to whether it was a weekday or
not. This allows analysis to occur of differing charging patterns
between weekdays and weekends, allowing for further accuracy in
determining the effects of electric vehicles on grid peaks.

In order to determine charging durations, rows were initially flagged as
``charging begins'' if the charging power was greater than zero and the
previous and following row's charging power were (respectively) equal to
zero and greater than zero. Similarly, rows were flagged as ``charge
ends'' if the charging power was greater than zero and the previous and
following row's charging power were (respectively) greater than zero and
equal to zero.

Using this method we obtained 7376 instances of charge beginning, and
7385 instances of charge ending. The additional 9 instances of the
charge ending than there are of the charge beginning may be due to the
first instance of data collection occurring during mid-charge for some
vehicles.

The charge duration was then calculated as being the time duration
between each pair of ``charge begins'' and ``charge ends'' flags.

Figure \ref{fig:durationHist} shows the overall distribution of all
charging sequences. Clearly there are very small and a few very large
values for both charging types.

\begin{figure}
\centering
\includegraphics{EVBB_report_files/figure-latex/durationHist-1.pdf}
\caption{\label{fig:durationHist}Duration of charging sequences}
\end{figure}

Table \ref{tab:durationDescTable} shows the overall distributions and
indicates the extent to which the means are skewed by the very small and
a few very large values shown in Figure \ref{fig:durationHist}.

\begin{table}[t]

\caption{\label{tab:durationDescTable}Duration of all charge sequences by charge type (minutes)}
\centering
\begin{tabular}{l|r|r|r|r|r}
\hline
chargeType & N & mean & median & min & max\\
\hline
Standard charging & 6983 & 101.24 & 3.72 & 0.27 & 1616.72\\
\hline
Fast charging & 392 & 38.00 & 12.48 & 0.32 & 8621.00\\
\hline
\end{tabular}
\end{table}

Figure \ref{fig:shortDuration} shows the distribution of very short
charging sequences. As we can see these appear to be generally less than
8 minutes in length for Standard Charges.

\begin{figure}
\centering
\includegraphics{EVBB_report_files/figure-latex/shortDuration-1.pdf}
\caption{\label{fig:shortDuration}Duration of charging sequences \textless{}
10 minutes}
\end{figure}

Table \ref{tab:durationDescTableReduced} shows the same descriptive
statistics but for all sequences of greater than 8 minute duration. Now
we can see that the mean and median durations for Standard Charge
sequences are closer to one another.

\begin{table}[t]

\caption{\label{tab:durationDescTableReduced}Duration of charge sequences > 8 minutes by charge type (minutes, )}
\centering
\begin{tabular}{l|r|r|r|r|r}
\hline
chargeType & N & mean & median & min & max\\
\hline
Standard charging & 2860 & 244.01 & 208.65 & 8.02 & 1616.72\\
\hline
Fast charging & 279 & 51.61 & 15.73 & 8.05 & 8621.00\\
\hline
\end{tabular}
\end{table}

Manual inspection of the data showed that these short-duration charging
``events'' generally occurred near the end of a longer-duration charging
event. It appeared that once the vehicle had reached its highest state
of charge, charging would intermittently stop and start again. This is
likely due to the behaviour of the charger once the battery was almost
full. As these can not be considered truly independent charging events,
they have been removed from the data for the analyses in Sections
\ref{keyFindings} and \ref{duration}.

In addition to the myriad ``small'' charging duration values, a small
amount of unreasonably long charging durations (longer than 100 hours
for standard charging or longer than 14 hours for fast charging) were
calculated. As these exceeded the expected charge durations of the most
high capacity vehicles currently available, they were assumed to be
anomalies and are not included in the analyses in Sections
\ref{keyFindings} and \ref{duration}.

Figure \ref{fig:longDuration} shows the distribution of charging
sequences with the excessively long or short events removed. These
charging durations appear more reasonable when considering standard
battery capacities and charging powers.

\begin{verbatim}
## `stat_bin()` using `bins = 30`. Pick better value with `binwidth`.
\end{verbatim}

\begin{figure}
\centering
\includegraphics{EVBB_report_files/figure-latex/longDuration-1.pdf}
\caption{\label{fig:longDuration}Duration of charging sequences with
unreasonably long or short values removed}
\end{figure}

\begin{verbatim}
## Saving 6.5 x 4.5 in image
## `stat_bin()` using `bins = 30`. Pick better value with `binwidth`.
\end{verbatim}

\section{Key Findings:}\label{keyFindings}

\begin{itemize}
\tightlist
\item
  \emph{Power supplied}: The median power supplied during a standard
  charging event was 1.78 kW. The mean was slightly higher at 2.12 kW.
  Fast charging observations had a median of 30.84 kW (mean = 30.68kW);
\item
  \emph{Charging duration}: Charging durations tended to fall into one
  of two groups. Longer `standard' charges had a median duration of 0.06
  hours and a mean duration of 1.69 hours. High power ``fast'' charge
  events had a median duration of 12.47 minutes and a mean duration of
  13.87 minutes.
\item
  \emph{Time of Day}: Standard charging events tended to begin around
  10pm, suggesting the drivers in our dataset utilise timers to take
  advantage of off-peak electricity. Fast charging events tended to
  begin at 11:30am on weekdays and 1pm during weekends.
\end{itemize}

\section{Observed demand}\label{observed-demand}

Figure \ref{fig:obsPower} shows the distribution of observed charging kW
demand by inferred charge type. This plot shows that fast charges are
relatively rare in the dataset whilst standard charges are much more
common, and are concentrated around 1.8kW, 3kW and 6kW.

\begin{verbatim}
## `stat_bin()` using `bins = 30`. Pick better value with `binwidth`.
\end{verbatim}

\begin{figure}
\centering
\includegraphics{EVBB_report_files/figure-latex/obsPower-1.pdf}
\caption{\label{fig:obsPower}Observed power demand distribution by charge
type where charging observed}
\end{figure}

75\% of standard charging observations were 1.47 kW or more but the
figure was 20.28 kW or more for fast charging

\section{Daily demand}\label{dailyDemand}

\begin{figure}
\centering
\includegraphics{EVBB_report_files/figure-latex/dailyPower-1.pdf}
\caption{\label{fig:dailyPower}Observed power demand distribution by day of
the week and charge type}
\end{figure}

Figure \ref{fig:dailyPower} shows the distribution of observed charging
kW demand by day of the week. We can see that fast charging varies
somewhat in daily demand, with weekends having a slightly lower demand
than weekdays, while standard charging is relatively constant across
days.

\section{Charging duration}\label{duration}

Figures \ref{fig:durationTimeBox} and \ref{fig:durationTimeMean} show
that the duration of standard charging events by event end time drops
significantly for events beginning ending 9:45am. This may indicate that
people are plugging in after returning home from a school run or other
morning activity, even though the battery is still close to full
capacity. It may also suggest that those who plug in shortly after
9:45am but do not have a high battery state of charge are only ``topping
up'', and take the vehicle out again before charging is fully complete.
Duration of fast charge events by event start time appear to be more
randomly distributed, although very few events were recorded between
midnight and 7am. This, along with the comparatively low number of
recorded fast charge events indicated in Fig. \ref{fig:obsPower}
suggests that drivers utilize fast charging only ``as necessary'' to
ensure they have enough battery capacity to complete their journey.

\begin{figure}
\centering
\includegraphics{EVBB_report_files/figure-latex/durationTimeBox-1.pdf}
\caption{\label{fig:durationTimeBox}Duration by time of charging start}
\end{figure}

\begin{figure}
\centering
\includegraphics{EVBB_report_files/figure-latex/durationTimeMean-1.pdf}
\caption{\label{fig:durationTimeMean}Mean duration (within quarter hours) by
time of charging end for sequences \textgreater{} 8 minutes}
\end{figure}

\begin{table}[t]

\caption{\label{tab:meanDurationTable}Mean duration of charge events by charge type}
\centering
\begin{tabular}{l|r|r|r|r|r}
\hline
chargeType & N & mean & median & min & max\\
\hline
Standard charging & 2860 & 244.00682 & 208.65000 & 8.016667 & 1616.717\\
\hline
Fast charging & 279 & 51.61231 & 15.73333 & 8.050000 & 8621.000\\
\hline
\end{tabular}
\end{table}

\section{State of charge}\label{SoC}

\begin{figure}
\centering
\includegraphics{EVBB_report_files/figure-latex/SoCplot1-1.pdf}
\caption{\label{fig:SoCplot1}Value of state of charge at beginning of
charge}
\end{figure}

\begin{verbatim}
## Saving 6.5 x 4.5 in image
\end{verbatim}

As can be seen in Figure \ref{fig:SoCplot1}, using the originally
defined ``charge begins'' data we have the majority of charges beginning
while the state of charge is above 90\%. This is likely due to the
manner in which the charger regularly turns off and on again near the
end of the charging cycle as described in Section \ref{cleaning}.

Figure \ref{fig:SoCplot2} shows the state of charge values when charge
begins but with state of charge greater than 90\% removed from the data.
The figure shows that many vehicles begin charging despite having
greater than 50\% charge remaining.

\begin{figure}
\centering
\includegraphics{EVBB_report_files/figure-latex/SoCplot2-1.pdf}
\caption{\label{fig:SoCplot2}Value of state of charge at beginning of charge
(\textgreater{}90\% values removed)}
\end{figure}

\begin{verbatim}
## Saving 6.5 x 4.5 in image
\end{verbatim}

\section{Time charging begins}\label{time-charging-begins}

After filtering out any data whereby charging begins while the state of
charge is greater than 90\% to account for battery `top-ups' (refer to
Section \ref{SoC}) we obtain the following figues.

\begin{verbatim}
## Picking joint bandwidth of 6060
\end{verbatim}

\begin{figure}
\centering
\includegraphics{EVBB_report_files/figure-latex/ggjoyplotTimeChargingBegins-1.pdf}
\caption{\label{fig:ggjoyplotTimeChargingBegins}Time charging begins}
\end{figure}

\begin{verbatim}
## <ggproto object: Class FacetGrid, Facet, gg>
##     compute_layout: function
##     draw_back: function
##     draw_front: function
##     draw_labels: function
##     draw_panels: function
##     finish_data: function
##     init_scales: function
##     map_data: function
##     params: list
##     setup_data: function
##     setup_params: function
##     shrink: TRUE
##     train_scales: function
##     vars: function
##     super:  <ggproto object: Class FacetGrid, Facet, gg>
\end{verbatim}

\begin{figure}
\centering
\includegraphics{EVBB_report_files/figure-latex/chargeBeginsWeekday-1.pdf}
\caption{\label{fig:chargeBeginsWeekday}Density plot of charging start times
during weekdays}
\end{figure}

\begin{verbatim}
## Saving 6.5 x 4.5 in image
\end{verbatim}

\begin{figure}
\centering
\includegraphics{EVBB_report_files/figure-latex/chargeBeginsWeekend-1.pdf}
\caption{\label{fig:chargeBeginsWeekend}Density plot of charging start times
during weekends}
\end{figure}

\begin{verbatim}
## Saving 6.5 x 4.5 in image
\end{verbatim}

Standard charging has a noticeably different profile to charging
patterns for fast charges. It suggests that it is common for plug-in
vehicle owners to charge overnight at home, and perhaps use the more
powerful public chargepoints to top up during the day.

Standard charging events most commonly began around 10pm during both
weekdays and weekends. As it seems unlikely that this is due to vehicle
drivers returning home at this hour, this effect may be due to drivers
setting the charger on a timer to take advantage of cheaper ``off-peak''
electricity times, which frequently begin around 10pm.

Fast charging events tended to begin at 11:30am on weekdays and 1pm
during weekends.

\begin{figure}
\centering
\includegraphics{EVBB_report_files/figure-latex/chargeTime-1.pdf}
\caption{\label{fig:chargeTime}Count of observed charging events by type,
day of week and time}
\end{figure}

Figure \ref{fig:chargeTime} shows the distribution of observed charging
by time of day and day of the week. Aggregating counts in this way
emphasises the times at which charging most commonly occurs and we can
see\ldots{}

Charging demand varies somewhat by time of day and day of the week.
Weekdays show \ldots{} whilst weekends show. Saturdays and Sundays vary
with\ldots{}

\begin{figure}
\centering
\includegraphics{EVBB_report_files/figure-latex/boxplotCharging-1.pdf}
\caption{\label{fig:boxplotCharging}Boxplot of daily standard charging
demand}
\end{figure}

\begin{figure}
\centering
\includegraphics{EVBB_report_files/figure-latex/plot3-1.pdf}
\caption{\label{fig:plot3}Boxplot of daily fast charging demand}
\end{figure}

\section{Implications to the electricity
grid}\label{implications-to-the-electricity-grid}

Under the assumption that later adopters of EVs follow the same charging
patterns as those of our data sample, we can make the following
predictions for an uptake of one EV per NZ household (1,729,300): *
\emph{Total electricity demand}:

\section{Summary}\label{summary}

Figure \ref{fig:SoCplot2} shows that many vehicles arrive home with
greater than 50\% charge remaining. This indicates that charging may be
delayed until early the following morning (during low aggregate
electricity demand) while providing enough ``back up'' state of charge
to allow for small evening trips if necessary. Alternatively, the
battery may be able to transfer energy to the home during the evening
grid peak as a form of demand response.

\section{References}\label{references}


\end{document}
